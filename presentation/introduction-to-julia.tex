\documentclass[aspectratio=169, handout]{beamer}
\usepackage[english]{babel}
\usepackage{booktabs,listings}
\usepackage[T1]{fontenc}
\usepackage[utf8]{inputenc}
\usepackage{pgfplots, enumitem}
\usepackage{amsmath, amssymb, bm}
\usepackage{minted, qrcode}
\AtBeginSection[]{
  {
    \setbeamertemplate{sidebar left}[NTNUverticalplain]%
    \begin{frame}
      \addtocounter{framenumber}{-1}
      \vfill
      \centering
      \begin{beamercolorbox}[sep=8pt,center,shadow=true,rounded=true]{title}
      \usebeamerfont{title}\insertsectionhead\par%
    \end{beamercolorbox}
    \vfill
    \end{frame}
  }
}
\usetheme[style=vertical, mathfont=serif]{NTNU}
\setlist[description]{font=\bfseries\color{NTNUBlue}, leftmargin=1em}
%
% Edit your meta data here
%
	\title{Introduction to \raisebox{-.18\height}{\includegraphics[width=1.75cm]{../img/julia-logo-color.png}}}
	\subtitle{Presentation and Workshop}
	\author[Ronny Bergmann]{\large{Ronny Bergmann}\\[\baselineskip]\ }
	\date[March 20, 2025]{Julia Users Group Trondheim\\[-.1\baselineskip]{\footnotesize and}\\[-.1\baselineskip]
    Department of Mathematical Sciences, NTNU.\\[\baselineskip]
    Trondheim,\hfill March 20, 2025.}
\begin{document}
	\maketitle
    \begin{frame}{Overview}
        \tableofcontents
    \end{frame}
    \section{What is Julia?}
    \begin{frame}{Goal: Scientific Computing \& Fast Prototyping}
    \end{frame}
    \begin{frame}{A short history}
    \end{frame}
    \begin{frame}{Resources}
        \begin{description}
            \item[Main homepage] \url{https://julialang.org}
            \item[Documentation] \url{https://docs.julialang.org/en/v1/}
            \item[Modern Julia Workflows] \url{https://modernjuliaworkflows.org/}
            \item[Discourse] \url{https://discourse.julialang.org}
            \item[JuliaHub] A webfrontend for the General Registry of Julia packages
            \url{https://juliahub.com/ui/Packages}
            \item[These slides] \url{https://github.com/Julia-Users-Trondheim/Intro-to-Julia/blob/main/presentation/introduction-to-julia.pdf}
        \end{description}
        \vspace{-\baselineskip}
        \begin{columns}
            \begin{column}{.71\textwidth}
            \end{column}
            \begin{column}{.29\textwidth}
                \qrcode[nolink,height=\textwidth]{https://github.com/Julia-Users-Trondheim/Intro-to-Julia/blob/main/presentation/introduction-to-julia.pdf}
            \end{column}
        \end{columns}
    \end{frame}
    \section{Installation \& REPL}
    \begin{frame}{Installation}
    \end{frame}
    \begin{frame}{Read-Eval-Print Loop (REPL)}
        The Julia command line is called \alert{REPL}.

        Quick commands
        \begin{description}
            \item[\^\ D] Quit
            \item[\^\ L] Clear console screen
            \item[Up Arrow] last command
        \end{description}
    \end{frame}
    \begin{frame}{REPL modes}
        Starting with special characters on REPL enters specific modes
        \begin{description}
            \item[?] help mode\\
            quick access to the documentation of a function
            \\[.5\baselineskip]
            \textbf{Example}:\\
            \mintinline{julia}!? sqrt! displays the help for the \mintinline{julia}|sqrt| function on REPL,
            \\
            see also the (HTML) documentation\\
            {\footnotesize\url{https://docs.julialang.org/en/v1/base/math/\#Base.sqrt-Tuple{Number}}}
            \\[-.66\baselineskip]
            \item[{]}] package mode
            \\ quick access to manage packages
            \item[;] shell mode
            \\ quick access to shell without exiting Julia,\\
            e.\,g.\ to change folders
        \end{description}
    \end{frame}
    \section{Main features}
    \begin{frame}{General philosophy}

    \end{frame}
    \begin{frame}{General code format}

    \end{frame}
    \begin{frame}{For-loops, while and such}

    \end{frame}
    \begin{frame}{Functions}

    \end{frame}
    \begin{frame}{structs – Data structures}

    \end{frame}
    \begin{frame}{Multiple Dispatch}

    \end{frame}
    \begin{frame}{Scripts}

    \end{frame}
    \section{Packages}
    \begin{frame}{Installing \& Using Pacakges}

    \end{frame}
    \begin{frame}{Package versions \& Updating}

    \end{frame}
    \begin{frame}{Package environments}

    \end{frame}
    \section{Pluto Notebooks}
    \begin{frame}{Pluto.jl – Motivation}
    \end{frame}
    \begin{frame}{Similarities \& differentes to Jupyter}
    \end{frame}
    \begin{frame}{Live Demo}
    \end{frame}
    \section{Workshop: Let's get you started with Julia!}
\end{document}
