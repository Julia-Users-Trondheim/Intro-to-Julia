\documentclass[aspectratio=169, 12pt]{beamer}
\usepackage[english]{babel}
\usepackage{booktabs,listings}
\usepackage[T1]{fontenc}
\usepackage[utf8]{inputenc}
\usepackage{pgfplots}
\usepackage{amsmath, amssymb, bm}
\usepackage{minted, qrcode}
\definecolor{juliared}{rgb}{0.796, 0.235, 0.2}
\definecolor{juliagreen}{rgb}{0.22, 0.596, 0.149}
\definecolor{juliablue}{rgb}{0.251, 0.388, 0.847}
\definecolor{juliapurple}{rgb}{0.584, 0.345, 0.698}
\setminted{style=default}
\setbeamersize{description width=0.3cm}
\AtBeginSection[]{
  {
    \setbeamertemplate{sidebar left}[NTNUverticalplain]%
    \begin{frame}
      \addtocounter{framenumber}{-1}
      \vfill
      \centering
      \begin{beamercolorbox}[sep=8pt,center,shadow=true,rounded=true]{title}
      \usebeamerfont{title}\insertsectionhead\par%
    \end{beamercolorbox}
    \vfill
    \end{frame}
  }
}
\usetheme[style=vertical, mathfont=serif]{NTNU}
%
% Edit your meta data here
%
	\title{Introduction to \raisebox{-.18\height}{\includegraphics[width=1.75cm]{../img/julia-logo-color.png}}}
	\subtitle{Presentation and Workshop}
	\author[Ronny Bergmann]{\large{Ronny Bergmann}\\[\baselineskip]\ }
	\date[March 20, 2025]{Julia Users Group Trondheim\\[-.1\baselineskip]{\footnotesize and}\\[-.1\baselineskip]
    Department of Mathematical Sciences, NTNU.\\[\baselineskip]
    Trondheim,\hfill March 20, 2025.}
\begin{document}
	\maketitle
    \begin{frame}{Overview}
        \tableofcontents
    \end{frame}
    \section{What is Julia?}
    \begin{frame}{Goal: Scientific Computing \& Fast Prototyping}
        In scientific computing we need
        \begin{itemize}
            \item high performance to tackle large scale problems
            \begin{itemize}
                \item[$\Rightarrow$] compiled languages (C/C++, Rust)
                \item all types are known at compile time
                \item static, hence maybe missing flexibility
            \end{itemize}
            \pause
            \item high-level dynamic languages (like Python, Matlab, R)
            \begin{itemize}
                \item[$\Rightarrow$] fast prototyping
                \item types have to be \emph{inferred} at runtime
                \item code is interpreted (slow)
            \end{itemize}
        \end{itemize}
        \pause
        \vspace{\baselineskip}
        \alert{Often:} Fast code is written in C/C++ and  is interfaced.
        \\[.5\baselineskip]
        $\Rightarrow$ new users might have to compile the C/C++ (e.g. MEX files)
    \end{frame}
    \begin{frame}{Combine both: Julia!}
        \alert{Julia} is
        \begin{itemize}
            \item dynamic with type inference
            \item just-in-time (JIT) compiled
            \item focusses on high-level numerical computing
        \end{itemize}
        \pause
        \vspace{\baselineskip}
        \textbf{\alert{A short history}}
        \begin{description}
            \item[2009] Adam Edelman starts the project with\\
            Jeff Bezanson, Stefan Karpinski, Viral B. Shah
            \item[2012] first public version
            \item[2018] Julia 1.0, i.e.\, no breaking releases since then
            \item[2024] Julia 1.11
        \end{description}
    \end{frame}
    \begin{frame}{Resources}
        \begin{description}
            \item[Main homepage] \url{https://julialang.org}
            \item[Documentation] \url{https://docs.julialang.org/en/v1/}
            \item[Modern Julia Workflows] \url{https://modernjuliaworkflows.org/}
            \item[Discourse] \url{https://discourse.julialang.org}
            \item[JuliaHub] webfrontend for the General Registry
            \url{https://juliahub.com/ui/Packages}
            \item[These slides]
        \end{description}
        \vspace{-1.25\baselineskip}
        \begin{columns}[T]
            \begin{column}{.075\textwidth}
            \end{column}
            \begin{column}{.64\textwidth}
                \ \\[.5\baselineskip]
                \url{https://github.com/Julia-Users-Trondheim/Intro-to-Julia/blob/main/presentation/introduction-to-julia.pdf}
            \end{column}
            \begin{column}{.33\textwidth}
                \qrcode[nolink,height=\textwidth]{https://github.com/Julia-Users-Trondheim/Intro-to-Julia/blob/main/presentation/introduction-to-julia.pdf}
            \end{column}
            \begin{column}{.005\textwidth}
            \end{column}
        \end{columns}
        \vspace{.75\baselineskip}
    \end{frame}
    \section{Installation \& REPL}
    \begin{frame}[fragile]{Installation}
        \begin{description}
            \item[Windows] Install Julia from the Microsoft Store by running this in the command prompt
            \begin{minted}{shell}
    winget install julia -s msstore
            \end{minted}
            \ \\
            \item[Mac OS / Linux] run the installer for example by
            \begin{minted}{shell}
    curl -fsSL https://install.julialang.org | sh
            \end{minted}
            \ldots or install \mintinline{shell}|juliaup| via your favourite package manager
        \end{description}
        \ \\
        We can take a closer look at your individual installation after this presentation in the workshop.
    \end{frame}
    \begin{frame}{Read-Eval-Print Loop (REPL)}
        The Julia command line is called \alert{REPL}.
        \begin{itemize}
            \item for fast computations
            \item easily define functions
            \item \mintinline{julia}|include("script.jl");| to run a script.
        \end{itemize}
        \vspace{\baselineskip}
        \pause
        \alert{\textbf{Quick commands}}
        \begin{description}
            \item[\^\ D] Quit
            \item[\^\ L] Clear console screen
            \item[Up Arrow] last command
        \end{description}
    \end{frame}
    \begin{frame}{REPL modes}
        Starting with special characters on REPL enters specific modes
        \begin{description}
            \item[?] help mode\\
            quick access to the documentation of a function
            \\[.5\baselineskip]
            \textbf{Example}:\\
            \mintinline{julia}!? sqrt! displays the help for the \mintinline{julia}|sqrt| function on REPL,
            \\
            see also the (HTML) documentation\\
            {\footnotesize\url{https://docs.julialang.org/en/v1/base/math/\#Base.sqrt-Tuple{Number}}}
            \\[-.66\baselineskip]
            \item[{]}] package mode
            \\ quick access to manage packages
            \item[;] shell mode
            \\ quick access to shell without exiting Julia,\\
            e.\,g.\ to change folders
        \end{description}
    \end{frame}
    \section{Main features}
    \begin{frame}{General philosophy}
        \begin{itemize}
            \item Write functions not scripts
            \item Julia has data types, but not objects
            \item write generic code “acting” on data
            \item no need to write “vectorized code”
            \item
        \end{itemize}
    \end{frame}
    \begin{frame}{General code format}
        \begin{itemize}
            \item Indentation with 4 spaces is recommended but not necessary
            \item blocks have an \mintinline{julia}|end|
            \item functions that modify their data should be named with an \mintinline{julia}|!|.
        \end{itemize}
    \end{frame}
    \begin{frame}{Control flow}

    \end{frame}
    \begin{frame}{Functions}

    \end{frame}
    \begin{frame}{Vectorized code vs. Broadcast}

    \end{frame}
    \begin{frame}{structs – Data structures}

    \end{frame}
    \begin{frame}{Multiple Dispatch}

    \end{frame}
    \begin{frame}{Functors}

    \end{frame}
    \begin{frame}{Scripts}

    \end{frame}
    \begin{frame}[fragile]{TLDR: Main differences to Python}
        \begin{itemize}
            \item \mintinline{julia}|for, if, while| etc. blocks are terminated by  \mintinline{julia}|end|
            \item indentation is nice, but not mandatory
            \item \alert{Julia is 1-indexed}
            \item Strings have single \mintinline{julia}|"quotation marks"|, multiline strings three
            \pause
            \item loops amd vectors are fast (no need for vectorized code)
            \item abstract arrays allow arbitrary indexing $\Rightarrow$ \mintinline{python}|a[-1]| is in Julia \mintinline{julia}|a[end-1]|
            \item Julias range \mintinline{julia}|1:5| includes the end and has the general form \mintinline{julia}|start:step:stop| (instead of \mintinline{python}|start:(stop+1):step|)
            \item the imaginary unit is \mintinline{julia}|im| (not \mintinline{python}|j|)
            \pause
            \item Matrix multiplication is \mintinline{julia}|A * B|, element wise multiplication \mintinline{julia}|A .* B|
            \item Julia has no objects/classes
        \end{itemize}
    \end{frame}
    \begin{frame}[fragile]{TLDR: Main differences to R}
        \begin{itemize}
            \item \mintinline{julia}|‘single’| quotation marks are for characters
            \item vectors are constructed with square brackets \mintinline{julia}|v = [1,2,3]|
            \item operations on vectors of different length are not allowed
            \item \mintinline{julia}|<-|, \mintinline{julia}|<<-| and \mintinline{julia}|->| are not assignment operators
            \item \mintinline{julia}|->| creates an anonymous function
            \pause
            \item matrix multiplication is just \mintinline{julia}|A * B|
            \item function arguments are not copied when calling a function
            \item \mintinline{julia}|1:5| is an \mintinline{julia}|AbstractRange|, use \mintinline{julia}|collect(1:5)| to create the vector
            \pause
            \item you do not need vectorization for performance
            \item logical indexing: in R \mintinline{R}|x[x>3]| has two alternatives in Julia
            \begin{itemize}
                \item \mintinline{julia}|x[ x .> 3]| (uses a temporary vector memory)
                \item \mintinline{julia}|filter(z->z>3, x)| might be nicer to read
                \item \mintinline{julia}|filter!(z->z>3, x)| updates \mintinline{julia}!x! inplace (avoids the temporary memory)
            \end{itemize}
        \end{itemize}
    \end{frame}
    \begin{frame}[fragile]{TLDR: Main differences to Matlab}
        \begin{itemize}
            \item array indexing uses square brackets \mintinline{julia}|A[i,j]|
            \item Arrays are not copied by default \mintinline{julia}|A=B| references the same, do \mintinline{julia}|A=copy(B)| for an actual copy
            \item \emph{similarly} function arguments are references, \alert{input variables can be modified}
            \item 1-dimensional vectors exist and are not \mintinline{julia}|Nx1| matrices
            \item \mintinline{julia}|42| is an integer, not a float, use \mintinline{julia}|42.0| for the float.
            \item \mintinline{julia}|A == B| does not return a matrix of booleans but \mintinline{julia}|true| or \mintinline{julia}|false|\\
            use \mintinline{julia}|A .== B| to get such a matrix
            \item dimensions are not “constant-broadcasted”:\\
            \begin{itemize}
                \item \mintinline{matlab}|[1:10] + [1:10]'| creates a $10\times 10$ matrix in Matlab
                \item \mintinline{julia}|[1:10] + [1:10]'| is a dimension mismatch,\\
                 because a column vector can not be added to a row vector
            \end{itemize}
        \end{itemize}
    \end{frame}
    \section{Packages}
    \begin{frame}{Installing \& Using Pacakges}

    \end{frame}
    \begin{frame}{Package versions \& Updating}

    \end{frame}
    \begin{frame}{Package environments}

    \end{frame}
    \section{Pluto Notebooks}
    \begin{frame}{Pluto.jl – Motivation}
    \end{frame}
    \begin{frame}{Similarities \& differentes to Jupyter}
    \end{frame}
    \begin{frame}{Live Demo}
    \end{frame}
    \section{Workshop: Let's get you started with Julia!}
\end{document}
