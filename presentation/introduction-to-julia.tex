\documentclass[aspectratio=169, 12pt]{beamer}
\usepackage[english]{babel}
\usepackage{booktabs,listings}
\usepackage[T1]{fontenc}
\usepackage[utf8]{inputenc}
\usepackage{pgfplots}
\usepackage{amsmath, amssymb, bm}
\usepackage{minted, qrcode}
\definecolor{juliared}{rgb}{0.796, 0.235, 0.2}
\definecolor{juliagreen}{rgb}{0.22, 0.596, 0.149}
\definecolor{juliablue}{rgb}{0.251, 0.388, 0.847}
\definecolor{juliapurple}{rgb}{0.584, 0.345, 0.698}
\setminted{style=default}
\setbeamersize{description width=0.3cm}
\AtBeginSection[]{
  {
    \setbeamertemplate{sidebar left}[NTNUverticalplain]%
    \begin{frame}
      \addtocounter{framenumber}{-1}
      \vfill
      \centering
      \begin{beamercolorbox}[sep=8pt,center,shadow=true,rounded=true]{title}
      \usebeamerfont{title}\insertsectionhead\par%
    \end{beamercolorbox}
    \vfill
    \end{frame}
  }
}
\usetheme[style=vertical, mathfont=serif]{NTNU}
%
% Edit your meta data here
%
	\title{Introduction to \raisebox{-.18\height}{\includegraphics[width=1.75cm]{../img/julia-logo-color.png}}}
	\subtitle{Presentation and Workshop}
	\author[Ronny Bergmann]{\large{Ronny Bergmann}\\[\baselineskip]\ }
	\date[March 20, 2025]{Julia Users Group Trondheim\\[-.1\baselineskip]{\footnotesize and}\\[-.1\baselineskip]
    Department of Mathematical Sciences, NTNU.\\[\baselineskip]
    Trondheim,\hfill March 20, 2025.}
\begin{document}
	\maketitle
    \begin{frame}{Overview}
        \tableofcontents
    \end{frame}
    \section{What is Julia?}
    \begin{frame}{Goal: Scientific Computing \& Fast Prototyping}
        In scientific computing we need
        \begin{itemize}
            \item high performance to tackle large scale problems
            \begin{itemize}
                \item[$\Rightarrow$] compiled languages (C/C++, Rust)
                \item all types are known at compile time
                \item static, hence maybe missing flexibility
            \end{itemize}
            \pause
            \item high-level dynamic languages (like Python, Matlab, R)
            \begin{itemize}
                \item[$\Rightarrow$] fast prototyping
                \item types have to be \emph{inferred} at runtime
                \item code is interpreted (slow)
            \end{itemize}
        \end{itemize}
        \pause
        \vspace{\baselineskip}
        \alert{Often:} Fast code is written in C/C++ and  is interfaced.
        \\[.5\baselineskip]
        $\Rightarrow$ new users might have to compile the C/C++ (e.g. MEX files)
    \end{frame}
    \begin{frame}{Combine both: Julia!}
        \alert{Julia} is
        \begin{itemize}
            \item dynamic with type inference
            \item just-in-time (JIT) compiled
            \item focusses on high-level numerical computing
        \end{itemize}
        \pause
        \vspace{\baselineskip}
        \textbf{\alert{A short history}}
        \begin{description}
            \item[2009] Adam Edelman starts the project with\\
            Jeff Bezanson, Stefan Karpinski, Viral B. Shah
            \item[2012] first public version
            \item[2018] Julia 1.0, i.e.\, no breaking releases since then
            \item[2024] Julia 1.11
        \end{description}
    \end{frame}
    \begin{frame}{Resources}
        \begin{description}
            \item[Main homepage] \url{https://julialang.org}
            \item[Documentation] \url{https://docs.julialang.org/en/v1/}
            \item[Modern Julia Workflows] \url{https://modernjuliaworkflows.org/}
            \item[Discourse] \url{https://discourse.julialang.org}
            \item[JuliaHub] webfrontend for the General Registry
            \url{https://juliahub.com/ui/Packages}
            \item[These slides]
        \end{description}
        \vspace{-1.25\baselineskip}
        \begin{columns}[T]
            \begin{column}{.075\textwidth}
            \end{column}
            \begin{column}{.64\textwidth}
                \ \\[.5\baselineskip]
                \url{https://github.com/Julia-Users-Trondheim/Intro-to-Julia/blob/main/presentation/introduction-to-julia.pdf}
            \end{column}
            \begin{column}{.33\textwidth}
                \qrcode[nolink,height=\textwidth]{https://github.com/Julia-Users-Trondheim/Intro-to-Julia/blob/main/presentation/introduction-to-julia.pdf}
            \end{column}
            \begin{column}{.005\textwidth}
            \end{column}
        \end{columns}
        \vspace{.75\baselineskip}
    \end{frame}
    \section{Installation \& REPL}
    \begin{frame}[fragile]{Installation}
        \begin{description}
            \item[Windows] Install Julia from the Microsoft Store by running this in the command prompt
            \begin{minted}{shell}
    winget install julia -s msstore
            \end{minted}
            \ \\
            \item[Mac OS / Linux] run the installer for example by
            \begin{minted}{shell}
    curl -fsSL https://install.julialang.org | sh
            \end{minted}
            \ldots or install \mintinline{shell}|juliaup| via your favourite package manager
        \end{description}
        \ \\
        We can take a closer look at your individual installation after this presentation in the workshop.
    \end{frame}
    \begin{frame}{Read-Eval-Print Loop (REPL)}
        The Julia command line is called \alert{REPL}.
        \begin{itemize}
            \item for fast computations
            \item easily define variables \& functions
            \item \mintinline{julia}|include("script.jl");| to run a script.
        \end{itemize}
        \vspace{\baselineskip}
        \pause
        \alert{\textbf{Quick commands}}
        \begin{description}
            \item[\^\ D] Quit
            \item[\^\ L] Clear console screen
            \item[Up Arrow] last command
        \end{description}
    \end{frame}
    \begin{frame}{REPL modes}
        Starting with special characters on REPL enters specific modes
        \begin{description}
            \item[?] help mode\\
            quick access to the documentation of a function
            \\[.5\baselineskip]
            \textbf{Example}:\\
            \mintinline{julia}!? sqrt! displays the help for the \mintinline{julia}|sqrt| function on REPL,
            \\
            see also the (HTML) documentation\\
            {\footnotesize\url{https://docs.julialang.org/en/v1/base/math/\#Base.sqrt-Tuple{Number}}}
            \\[-.66\baselineskip]
            \item[{]}] package mode
            \\ quick access to manage packages
            \item[;] shell mode
            \\ quick access to shell without exiting Julia,\\
            e.\,g.\ to change folders
        \end{description}
    \end{frame}
    \section{Main features}
    \begin{frame}{General philosophy \& Code format}
        \alert{\textbf{Philosophy}}
        \begin{itemize}
            \item Write functions not scripts
            \item Julia has data types, but not objects
            \item write generic code “acting” on data
            \item no need to write “vectorized code”
            \item avoid global variables
        \end{itemize}
        \vspace{\baselineskip}
        \pause
        \alert{\textbf{Format}}
        \begin{itemize}
            \item blocks have an \mintinline{julia}|end|
            \item Indentation with 4 spaces is recommended but not necessary
            \item functions that modify their data should be named with an \mintinline{julia}|!|.
        \end{itemize}
    \end{frame}
    \begin{frame}[fragile]{Prequel: Packages \& Pluto Notebooks}
        A \alert{Package} is a \alert{module} (namespace) providing additional functionality.

        \begin{itemize}[<+->]
            \item To install one for our demos use the package mode
            \begin{minted}{julia-repl}
] add Pluto
            \end{minted}
            This has only to be done once.
            \item To load a package after starting Julia,
            use the \mintinline{julia}{using} keyword
            \begin{minted}{julia-repl}
using Pluto
            \end{minted}
            \item we can call a function from the package always by
            \begin{minted}{julia-repl}
Pluto.run()
            \end{minted}
            \item the last two can be done in one line, when using \mintinline{julia}|;| as a divider
            \begin{minted}{julia-repl}
using Pluto; Pluto.run()
            \end{minted}
        \end{itemize}
        \pause
        We will continue command demos in the \href{https://plutojl.org}{Pluto notebook}
        \\\hfill (similar to a Jupyter notebook, but with a persistent state)
    \end{frame}
    \begin{frame}[fragile]{Control flow I: for \& while}
        \begin{columns}[T]
            \begin{column}{.5\textwidth}
                \alert{Iterate} with for-loops
                \begin{minted}{julia}
for i=1:4
    print(i," ")
end # prints "1 2 3 4"
                \end{minted}
                \pause
                Combine several (and use $\in$)
                \begin{minted}[escapeinside=||,mathescape=true]{julia}
for i |$\in$| 1:3, j |$\in$| 1:2
    print(i,"×",j,", ")
end # prints 1$\times$1, 1$\times$2, ...
                \end{minted}
                \pause
                Or through several of same length
                \begin{minted}[escapeinside=||,mathescape=true]{julia}
for (i,j) |$\in$| zip(1:4, 5:8)
    print(i,"|",j," ")
end # prints 1|5 2|6 3|7 4|8
                \end{minted}
            \end{column}
            \pause
            \begin{column}{.5\textwidth}
                or as a \alert{comprehension} for vectors
                \begin{minted}[escapeinside=||,mathescape=true]{julia}
x = [3*s for s |$\in$| 1:3 ]
                \end{minted}
                creates \mintinline{julia}|[3, 6, 9]|
                \\[2.5\baselineskip]
                Loops with “unknown end”
                \begin{minted}{julia}
i = 1;
# do as long as i <= 4
while i <= 4
    print(i," ");
    i += 1
end # also prints "1 2 3 4"
                \end{minted}
            \end{column}
        \end{columns}
    \end{frame}
    \begin{frame}[fragile]{Control flow II: Conditionals}
        \alert{Conditionals} require an expression that evaluates to a \mintinline{julia}|Bool|. Then
        \begin{minted}{julia}
if (x > 3) || (z < 2) # brackets (x > 3) are optional
    print("x is at least 3")
else
    print("x is 3 or less")
end
        \end{minted}
        \pause
        There is \alert{lazy evaluation}: the second parts of
        \begin{minted}{julia}
 (x > 4) && print("x > 4")
(x <= 4) || print("x > 4")
        \end{minted}
        are only called/evaluated if $x > 4$.
        \\[\baselineskip]
        \pause
        Conditionals can be used inline with
        \begin{minted}{julia}
y = (x > 4) ? 1 : 3*x
        \end{minted}
    \end{frame}
    \begin{frame}[fragile]{Defining functions}
        \begin{minted}{julia}
"""
    phase(z)

Compute the phase of a complex number z
"""
function phase(z)
    return atan(imag(z), real(z))
end
        \end{minted}
        \begin{itemize}[<+->]
            \item \alert{naming convention} \mintinline{julia}|snake_case|
            \item (multiline) \mintinline{julia}|"String"| upfront: doc-string, may use  Markdown
            \item specify type: \mintinline{julia}|z::Number| (\alert{avoid} overtyping as \mintinline{julia}{::Float64})
            \item (last) \mintinline{julia}|return| optional, but reommended\\
            \hfill(last evaluated expression returned)
        \end{itemize}
    \pause
    Shorter form
    \begin{minted}{julia}
magnitude(z) = sqrt(imag(z)^2+real(z)^2)
    \end{minted}
    \end{frame}
    \begin{frame}[fragile]{More on functions I: positional \& keyword args}
        \begin{itemize}
            \item \alert{positional optional} parameters: providing defaults
        \begin{minted}{julia}
f(a, b=2, c=3) = a*exp(b/c)
f(1) #equals f(1,2,3)
f(1,3) #equals f(1,3,3)
f(1,3,5) #equals f(1,3,5)
        \end{minted}
        \item short to write, \alert{but} to set \mintinline{julia}|c|, you always have to provide \mintinline{julia}|b|
            \pause
            \item \alert{keyword arguments} are provided after a \mintinline{julia}|;|
        \begin{minted}{julia}
g(a; b=2, c=3) = a*exp(b/c)
g(1; b=3) #equals g(1; b=3, c=3)
g(1; c=5) #equals g(1; b=2, c=5)
        \end{minted}
            \item must variable name to set a value, order is \alert{not} important.
            \pause
            \item in \mintinline{julia}|g(1; b=4, b=3)| the last one “wins”, so \mintinline{julia}|b| is 3.
            \pause
        \item You can “collect and pass on”:
            \begin{itemize}
                \item \mintinline{julia}|h1(args...) = f(1, args...)|
                \item \mintinline{julia}|h2(; kwargs...) = g(1; kwargs...)|
                \item or combine both as \mintinline{julia}|h3(args...; kwargs...) = #def here|
            \end{itemize}
        \end{itemize}
    \end{frame}
    \begin{frame}[fragile]{More on functions II: broadcast and mutation}
        \begin{itemize}[<+->]
            \item functions are first-class objects (like variables)
            \item anonymous function \mintinline{julia}|(x,y) -> x^y| e.g., to pass as parameter
            \item \alert{Broadcast}: apply \mintinline{julia}|phase(z)| to a whole vector
            \begin{minted}{julia}
Z = [1.0im, 2.0, 1.0 + 0.2im]
            \end{minted}
            by adding a \mintinline{julia}|.| after the function name: \mintinline{julia}|phase.(Z)|
            \item broadcast with multiple vectors
            \begin{minted}{julia}
X = [0.1, 0.2, 0.3]; Y = [1.0, 2.0, 3.0]
X.^Y # same: [X[i]^Y[i] for i=1:3] or [0.1, 0.04, 0.027]
            \end{minted}
            \item functions can modify their input
            \begin{minted}{julia}
function add_scalar!(X, v)
    X .+= v  # X an array, v a scalar: add to every entry
    return X # the X we got passed is now changed
end
            \end{minted}
            \alert{Convention:} modifying function names end with \mintinline{julia}|!|\\
            and return the modified variable.
        \end{itemize}
    \end{frame}
    \begin{frame}[fragile]{Data structures}
        \vspace{-.25\baselineskip}
        \begin{columns}[T]
            \begin{column}{.65\textwidth}
                \alert{Abstract types} to build a type hierarchy:
                \begin{minted}{julia}
abstract type ExperimentData end
                \end{minted}
            \end{column}
            \pause
            \begin{column}{.33\textwidth}
            \alert{naming convention:}\\ Types are
            \mintinline{julia}|CamelCase|.
            \end{column}
        \end{columns}
        \vspace{\baselineskip}
        \pause
        \begin{columns}[T]
            \begin{column}{.49\textwidth}
                \textbf{Variant I.} default: immutable
                \begin{minted}{julia}
struct TimeSeries <: ExperimentData
    name::String
    data::Vector
end # default constructor:
ts = TimeSeries("A", [1,2,3])
                \end{minted}
            \end{column}
            \begin{column}{.5\textwidth}
                \vspace{-.25\baselineskip}
                \pause
                \begin{itemize}[<+->]
                    %\item avoid untyped fields\\
                    %they get type \mintinline{julia}|::Any|
                    \item fields can not be (ex)changed:\\
                    \mintinline{julia}|ts.name="B"| and
                    \mintinline{julia}|ts.data=[4,5]| error.\\[.25\baselineskip]
                    \item\alert{but} \mintinline{julia}|ts.data[2]=4| works
                    \\(modified in-place)
                    \item more efficient
                \end{itemize}
            \end{column}
        \end{columns}
        \pause
        \vspace{1\baselineskip}
        \begin{columns}[T]
            \begin{column}{.51\textwidth}
                \textbf{Variant II.} mutable – reassign fields:
                \begin{minted}{julia}
mutable struct Measurement <: ExperimentData
    name::String
    value::Float64
end # same constructor
m = Measurement("B", 3.1415)
                \end{minted}
            \end{column}
            \begin{column}{.48\textwidth}
                \pause
                \ \\[1.5\baselineskip]
                \begin{itemize}[<+->]
                    \item \mintinline{julia}|m.name="B"; m.value=4.5|\\
                    both work (if same type)
                    \item slightly less efficient
                \end{itemize}
            \end{column}
        \end{columns}
    \end{frame}
    \begin{frame}[fragile]{Paremetric types \& functions}
        \begin{itemize}[<+->]
            \item ensure two fields have \alert{exactly the same type}
            \item to avoid abstract types in concrete instances (reduce performance)
            \item stay flexible to for new use cases
        \end{itemize}
        \pause
        \begin{minted}{julia}
mutable struct TimeSeries2{T} <: ExperimentData
    param::T        # maybe some concentration
    data::Vector{T} # actually parametrized by element-type
end # Constructor now maybe a bit clumsy:
ts2 = TimeSeries2{Float64}(3.1415, [1.2, 1.3])
        \end{minted}
        \pause
        \begin{itemize}
            \item makes the previous (implicit) \mintinline{julia}|Vector{Any}| to a concrete type
            \pause
            \item nicer constructor: Define a \alert{parametric function}
        \end{itemize}
        \begin{minted}{julia}
function TimeSeries2(c::T, v::Vector{T}) where {T}
    return TimeSeries2{T}(c, v)
end # Then we have back
ts2 = TimeSeries2(3.1415, [1.2, 1.3])
        \end{minted}
    \end{frame}
    \begin{frame}[fragile]{Multiple Dispatch}
        \begin{columns}[T]
            \begin{column}{.37\textwidth}
                \textbf{Dispatch}: “finding” the “best fitting version”
                of a function.
                \pause For
                \\
                \begin{minted}{julia}
f(x) = "A"
f(x::Number) = "B"
f(x::Float64) = "C"
                \end{minted}
                \pause
                \vspace{\baselineskip}
                We get that
                \begin{minted}{julia}
f.(["a", 1, 1.0im, 2.0])
                \end{minted}
                is \mintinline{julia}|["A", "B", "B", "C"]|
                \pause
                \\[\baselineskip]
                $\Rightarrow$ dispatch to\\
                “most fitting”\\
                \alert{method} of a \alert{function}
            \end{column}
            \begin{column}{.62\textwidth}
                \vspace{-.25\baselineskip}
                \begin{minted}{julia}
function g(a::Number, t::TimeSeries)
  TimeSeries(t.name, a .* t.data)
end
function g(a::String, t::TimeSeries)
  TimeSeries("$(a) $(t.name)", t.data)
end
function g(a::Number, ts::TimeSeries2)
  TimeSeries2(a*t.param, a .* t)
end
                \end{minted}
                \pause
                \alert{\textbf{Avoid}} ambiguities. Defining
                \begin{minted}{julia}
g(a::Float64, b) = 2*a+b
g(a, b::Float64) = a+2*b
                \end{minted}
                makes \mintinline{julia}|g(1.0,2.0)|
                \alert{ambiguous}.
                \pause Resolve by
                \begin{minted}{julia}
g(a::Float64, b::Float64) = 2*a + 2*b
                \end{minted}
            \end{column}
        \end{columns}
    \end{frame}
    \begin{frame}[fragile]{Operators are Functions}
        Operators like \mintinline{julia}|+, *, ^| are \alert{functions}. Add a method to \mintinline{julia}|+| via
        \begin{minted}{julia}
function Base.:+(t::TimeSeries, s::TimeSeries)
    if !(length(t.data)==length(s.data))
        error("Time series not of same length")
    end
    return TimeSeries(
        "$(t.name) and $(s.name)",
        t.data .+ s.data
    )
end
        \end{minted}
%        The \mintinline{julia}|$(a)| is \alert{string interpolation}.
        \pause
        Then
        \begin{minted}{julia}
u = TimeSeries("A", [1,2]) + TimeSeries("B", [3,4])
        \end{minted}
        returns \mintinline{julia}|TimeSeries("A and B", [4, 6])|.
        \\[.5\baselineskip]\pause
        To ensure same type parameter, define a function with
        \begin{minted}{julia}
Base.:+(t::TimeSeries2{T}, s::TimeSeries2{T}) where {T}
        \end{minted}
    \end{frame}
    \begin{frame}[fragile]{Functors: function-like structures}
    Consider (actually taken from the Julia documentation)
    \begin{minted}{julia}
struct Polynomial{R}
    coeffs::Vector{R}
end
    \end{minted}
    \pause
    We can turn a \mintinline{julia}|Polynomial| into a function as well definiing
    \begin{minted}{julia}
function (p::Polynomial)(x)
    v = p.coeffs[end] # Horner Schema, (a_2x + a_1)x + a_0
    for i = (length(p.coeffs)-1):-1:1
        v = v*x + p.coeffs[i]
    end
    return v
end
    \end{minted}
    \pause
    For \mintinline{julia}|p = Polynomial([1, 10, 100]); p(3)| we get $100\cdot3^2 + 10\cdot3 + 1 = 931$
    \end{frame}
    \begin{frame}[fragile]{TLDR: Main differences to Python}
        \begin{itemize}
            \item \mintinline{julia}|for, if, while| etc. blocks are terminated by  \mintinline{julia}|end|
            \item indentation is nice, but not mandatory
            \item \alert{Julia is 1-indexed}
            \item Strings have single \mintinline{julia}|"quotation marks"|, multiline strings three
            \pause
            \item loops amd vectors are fast (no need for vectorized code)
            \item abstract arrays allow arbitrary indexing $\Rightarrow$ \mintinline{python}|a[-1]| is in Julia \mintinline{julia}|a[end-1]|
            \item Julias range \mintinline{julia}|1:5| includes the end and has the general form \mintinline{julia}|start:step:stop| (instead of \mintinline{python}|start:(stop+1):step|)
            \item the imaginary unit is \mintinline{julia}|im| (not \mintinline{python}|j|)
            \pause
            \item Matrix multiplication is \mintinline{julia}|A * B|, element wise multiplication \mintinline{julia}|A .* B|
            \item Julia has no objects/classes
        \end{itemize}
    \end{frame}
    \begin{frame}[fragile]{TLDR: Main differences to R}
        \begin{itemize}
            \item \mintinline{julia}|‘single’| quotation marks are for characters
            \item vectors are constructed with square brackets \mintinline{julia}|v = [1,2,3]|
            \item operations on vectors of different length are not allowed
            \item \mintinline{julia}|<-|, \mintinline{julia}|<<-| and \mintinline{julia}|->| are not assignment operators
            \item \mintinline{julia}|->| creates an anonymous function
            \pause
            \item matrix multiplication is just \mintinline{julia}|A * B|
            \item function arguments are not copied when calling a function
            \item \mintinline{julia}|1:5| is an \mintinline{julia}|AbstractRange|, use \mintinline{julia}|collect(1:5)| to create the vector
            \pause
            \item you do not need vectorization for performance
            \item logical indexing: in R \mintinline{R}|x[x>3]| has two alternatives in Julia
            \begin{itemize}
                \item \mintinline{julia}|x[ x .> 3]| (uses a temporary vector memory)
                \item \mintinline{julia}|filter(z->z>3, x)| might be nicer to read
                \item \mintinline{julia}|filter!(z->z>3, x)| updates \mintinline{julia}!x! inplace (avoids the temporary memory)
            \end{itemize}
        \end{itemize}
    \end{frame}
    \begin{frame}[fragile]{TLDR: Main differences to Matlab}
        \begin{itemize}
            \item array indexing uses square brackets \mintinline{julia}|A[i,j]|
            \item Arrays are not copied by default \mintinline{julia}|A=B| references the same, do \mintinline{julia}|A=copy(B)| for an actual copy
            \item \emph{similarly} function arguments are references, \alert{input variables can be modified}
            \item 1-dimensional vectors exist and are not \mintinline{julia}|Nx1| matrices
            \item \mintinline{julia}|42| is an integer, not a float, use \mintinline{julia}|42.0| for the float.
            \item \mintinline{julia}|A == B| does not return a matrix of booleans but \mintinline{julia}|true| or \mintinline{julia}|false|\\
            use \mintinline{julia}|A .== B| to get such a matrix
            \item dimensions are not “constant-broadcasted”:\\
            \begin{itemize}
                \item \mintinline{matlab}|[1:10] + [1:10]'| creates a $10\times 10$ matrix in Matlab
                \item \mintinline{julia}|[1:10] + [1:10]'| is a dimension mismatch,\\
                 because a column vector can not be added to a row vector
            \end{itemize}
        \end{itemize}
    \end{frame}
    \section{Packages}
    \begin{frame}[fragile]{Namespaces \& Modules}
        \begin{minted}{julia}
module MyModule #Same naming convention as types: CamelCase
    f(x) = x^2         # is exported
    struct MyField end # is not exported
    export f
end
        \end{minted}
        \begin{itemize}[<+->]
            \item introduces a namespace, loaded with \mintinline{julia}|using .MyModule|
            \\\hfill{\footnotesize (the \mintinline{julia}|.| necessary for modules defined in scripts/REPL)}
            \item a module can internally also use other (dependent) packages.
            \item anything it \mintinline{julia}|export|s is available in \alert{global} namespace
            \item other functions/structs via \mintinline{julia}|MyModule.local_function|
            \item[!] if two modules \mintinline{julia}|A| and \mintinline{julia}|B| exort \mintinline{julia}|f|, one also has to use
            \mintinline{julia}|A.f| and \mintinline{julia}|B.f|
            \\ or specify which one to use with \mintinline{julia}|using A: f|
            \item Default packages are among others \mintinline{julia}|Base| (loaded on start)\\
            \mintinline{julia}|LinearAlgebra|,
            \mintinline{julia}|Random,|
            \mintinline{julia}|Statistics|, ...
        \end{itemize}
    \end{frame}
    \begin{frame}[fragile]{Installing \& Using Packages}
        \begin{itemize}[<+->]
            \item modules that come from a \alert{Registry}, package manager: \mintinline{julia}|Pkg.jl|
            \item default: \url{https://github.com/JuliaRegistries/General}
            \item Shortcut: \alert{Package mode} in REPL; Start command with \mintinline{julia}|] |
            \item \mintinline{julia}|] add PackageName| installs a package
            \begin{itemize}
                \item including all packages \mintinline{julia}{PackageName} \alert{depends} on.
                \item resolves versions to “fit” to all already installed ones
            \end{itemize}
            \item \mintinline{julia}|] status| lists all installed packages with their versions
            \item \mintinline{julia}|] update| update all packages to newest version
        \end{itemize}
        \pause
        After a package is installed, it can be used with
        \begin{minted}{julia}
using PackageName, PackageA, PackageB,
        \end{minted}
    \end{frame}
    \begin{frame}{Package environments}
        \begin{itemize}
            \item in package mode: \mintinline{shell}|(@v1.11) pkg>| refers to\\
            current environment: by default the global one
            \item an \alert{environment} is a set of packages and their versions
            \item use \mintinline{julia}|] activate Name| to activate a new environment
            \item use \mintinline{julia}|] activate .| to turn the current folder into an environment.
            \begin{itemize}[<+->]
                \item[$\Rightarrow$] This is easy to activate for a set of scripts
                \item[$\Rightarrow$] reproducible: in the environment,
                we always have the same packages/package versions
                \item[$\Rightarrow$] file \mintinline{julia}|Project.toml| allows others to
                activate and \mintinline{julia}|] instantiate| (install its packages) on other machines as well
                \item even safer: \mintinline{julia}|Manifest.toml| all packages and their dependencies in \alert{exact versions} resolved
            \end{itemize}
        \end{itemize}
        \pause
        \alert{\textbf{$\Rightarrow$}} \textbf{Reproducible} environment / setup to run your experiments in
    \end{frame}
    \section{Pluto Notebooks}
    \begin{frame}[fragile]{The Julia package Pluto.jl}
        \hfill\href{https://plutojl.org}{plutojl.org}
        \begin{itemize}
            \item browser-based code development
            \item purely Julia based
            \pause
            \item only code-cells with \alert{one} command per cell
            \begin{itemize}
                \item use \mintinline{julia}|begin ... end| block to wrap multiple commands
                \item Markdown cell: a \mintinline{julia}|md"..."| (\mintinline{julia}|md"""..."""| multiline) string
            \end{itemize}
            \item execute cell by \mintinline{shell}|Shift+Enter| or saving the file.
            \pause
            \item For Markdown or long, technical code cells: hide code.
            \pause
            \item Live-docs – display the documentation of current function
            \pause
            \item similar to Mathematica or Jupyter notebooks
        \end{itemize}
        On terminal \mintinline{julia}|using Pluto; Pluto.run();| to start the webserver.
    \end{frame}
    \begin{frame}{Notable differentes to Jupyter}
        \begin{itemize}[<+->]
            \item the Pluto notebook is saved as a script \mintinline{julia}|nootebook.jl|
            \begin{itemize}
                \item[$\Rightarrow$] it can also be run using \mintinline{julia}|include("notebook.jl")| on REPL
                \item[{$\ominus$}] output of cells is not saved to file
                \item[{$\oplus$}] the source code file fits well into version management like \mintinline{shell}|git|
            \end{itemize}
            \item a Pluto notebook “knows its used packages versions”:
            \begin{itemize}[<+->]
                \item it opens an own environment on start
                \item keeps track of all (exact!) versions of the installed packages
                \item[$\oplus$] it is running reproducibly, even on other peoples computers!
            \end{itemize}
            \item The pluto notebook has a \alert{persistent state}
            \begin{itemize}
                \item internally keeps track which cells depend on others
                \item[$\Rightarrow$] changing a parameter updates \alert{all} dependent cells
                \item[$\oplus$] all cells always reflect the current global state of code
                \item[$\Rightarrow$] you never have to remember to “execute cells in right order”
            \end{itemize}
        \end{itemize}
    \end{frame}
    \begin{frame}{Live Demo}
    \end{frame}
    \begin{frame}{Further topics}
        \begin{itemize}
            \item further default data structures
            \begin{itemize}
                \item \mintinline{julia}|Dict| dictionaries
                \item \mintinline{julia}|NamedTuple|s as “lightweight, flexible” struct
                \item \mintinline{julia}|IO| reading/writing files
                \item further packages from the Standard Library
            \end{itemize}
            \item \mintinline{julia}|@macro|s – rewriting code
            \item VS Code extension \& the debugger
            \item specific packages for your concrete problems
            \item \mintinline{julia}|Test.jl| and running tests on your own package
            \item \mintinline{julia}|Documenter.jl| and creating a documentation for your own package
            \item \alert{package extensions} and weak dependencies
        \end{itemize}
    \end{frame}
    \begin{frame}{Thanks for your attention!}
        Are there (further) questions?
    \end{frame}
    \section{Workshop: Let's get you started with Julia!}
\end{document}
