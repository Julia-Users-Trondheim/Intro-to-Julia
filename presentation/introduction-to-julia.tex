\documentclass[aspectratio=169, 12pt]{beamer}
\usepackage[english]{babel}
\usepackage{booktabs,listings}
\usepackage[T1]{fontenc}
\usepackage[utf8]{inputenc}
\usepackage{pgfplots}
\usepackage{amsmath, amssymb, bm}
\usepackage{minted, qrcode}
\definecolor{juliared}{rgb}{0.796, 0.235, 0.2}
\definecolor{juliagreen}{rgb}{0.22, 0.596, 0.149}
\definecolor{juliablue}{rgb}{0.251, 0.388, 0.847}
\definecolor{juliapurple}{rgb}{0.584, 0.345, 0.698}
\setminted{style=default}
\setbeamersize{description width=0.3cm}
\AtBeginSection[]{
  {
    \setbeamertemplate{sidebar left}[NTNUverticalplain]%
    \begin{frame}
      \addtocounter{framenumber}{-1}
      \vfill
      \centering
      \begin{beamercolorbox}[sep=8pt,center,shadow=true,rounded=true]{title}
      \usebeamerfont{title}\insertsectionhead\par%
    \end{beamercolorbox}
    \vfill
    \end{frame}
  }
}
\usetheme[style=vertical, mathfont=serif]{NTNU}
%
% Edit your meta data here
%
	\title{Introduction to \raisebox{-.18\height}{\includegraphics[width=1.75cm]{../img/julia-logo-color.png}}}
	\subtitle{Presentation and Workshop}
	\author[Ronny Bergmann]{\large{Ronny Bergmann}\\[\baselineskip]\ }
	\date[March 20, 2025]{Julia Users Group Trondheim\\[-.1\baselineskip]{\footnotesize and}\\[-.1\baselineskip]
    Department of Mathematical Sciences, NTNU.\\[\baselineskip]
    Trondheim,\hfill March 20, 2025.}
\begin{document}
	\maketitle
    \begin{frame}{Overview}
        \tableofcontents
    \end{frame}
    \section{What is Julia?}
    \begin{frame}{Goal: Scientific Computing \& Fast Prototyping}
        In scientific computing we need
        \begin{itemize}
            \item high performance to tackle large scale problems
            \begin{itemize}
                \item[$\Rightarrow$] compiled languages (C/C++, Rust)
                \item all types are known at compile time
                \item static, hence maybe missing flexibility
            \end{itemize}
            \pause
            \item high-level dynamic languages (like Python, Matlab, R)
            \begin{itemize}
                \item[$\Rightarrow$] fast prototyping
                \item types have to be \emph{inferred} at runtime
                \item code is interpreted (slow)
            \end{itemize}
        \end{itemize}
        \pause
        \vspace{\baselineskip}
        \alert{Often:} Fast code is written in C/C++ and  is interfaced.
        \\[.5\baselineskip]
        $\Rightarrow$ new users might have to compile the C/C++ (e.g. MEX files)
    \end{frame}
    \begin{frame}{Combine both: Julia!}
        \alert{Julia} is
        \begin{itemize}
            \item dynamic with type inference
            \item just-in-time (JIT) compiled
            \item focusses on high-level numerical computing
        \end{itemize}
        \pause
        \vspace{\baselineskip}
        \textbf{\alert{A short history}}
        \begin{description}
            \item[2009] Adam Edelman starts the project with\\
            Jeff Bezanson, Stefan Karpinski, Viral B. Shah
            \item[2012] first public version
            \item[2018] Julia 1.0, i.e.\, no breaking releases since then
            \item[2024] Julia 1.11
        \end{description}
    \end{frame}
    \begin{frame}{Resources}
        \begin{description}
            \item[Main homepage] \url{https://julialang.org}
            \item[Documentation] \url{https://docs.julialang.org/en/v1/}
            \item[Modern Julia Workflows] \url{https://modernjuliaworkflows.org/}
            \item[Discourse] \url{https://discourse.julialang.org}
            \item[JuliaHub] webfrontend for the General Registry
            \url{https://juliahub.com/ui/Packages}
            \item[These slides]
        \end{description}
        \vspace{-1.25\baselineskip}
        \begin{columns}[T]
            \begin{column}{.075\textwidth}
            \end{column}
            \begin{column}{.64\textwidth}
                \ \\[.5\baselineskip]
                \url{https://github.com/Julia-Users-Trondheim/Intro-to-Julia/blob/main/presentation/introduction-to-julia.pdf}
            \end{column}
            \begin{column}{.33\textwidth}
                \qrcode[nolink,height=\textwidth]{https://github.com/Julia-Users-Trondheim/Intro-to-Julia/blob/main/presentation/introduction-to-julia.pdf}
            \end{column}
            \begin{column}{.005\textwidth}
            \end{column}
        \end{columns}
        \vspace{.75\baselineskip}
    \end{frame}
    \section{Installation \& REPL}
    \begin{frame}[fragile]{Installation}
        \begin{description}
            \item[Windows] Install Julia from the Microsoft Store by running this in the command prompt
            \begin{minted}{shell}
    winget install julia -s msstore
            \end{minted}
            \ \\
            \item[Mac OS / Linux] run the installer for example by
            \begin{minted}{shell}
    curl -fsSL https://install.julialang.org | sh
            \end{minted}
            \ldots or install \mintinline{shell}|juliaup| via your favourite package manager
        \end{description}
        \ \\
        We can take a closer look at your individual installation after this presentation in the workshop.
    \end{frame}
    \begin{frame}{Read-Eval-Print Loop (REPL)}
        The Julia command line is called \alert{REPL}.
        \begin{itemize}
            \item for fast computations
            \item easily define variables \& functions
            \item \mintinline{julia}|include("script.jl");| to run a script.
        \end{itemize}
        \vspace{\baselineskip}
        \pause
        \alert{\textbf{Quick commands}}
        \begin{description}
            \item[\^\ D] Quit
            \item[\^\ L] Clear console screen
            \item[Up Arrow] last command
        \end{description}
    \end{frame}
    \begin{frame}{REPL modes}
        Starting with special characters on REPL enters specific modes
        \begin{description}
            \item[?] help mode\\
            quick access to the documentation of a function
            \\[.5\baselineskip]
            \textbf{Example}:\\
            \mintinline{julia}!? sqrt! displays the help for the \mintinline{julia}|sqrt| function on REPL,
            \\
            see also the (HTML) documentation\\
            {\footnotesize\url{https://docs.julialang.org/en/v1/base/math/\#Base.sqrt-Tuple{Number}}}
            \\[-.66\baselineskip]
            \item[{]}] package mode
            \\ quick access to manage packages
            \item[;] shell mode
            \\ quick access to shell without exiting Julia,\\
            e.\,g.\ to change folders
        \end{description}
    \end{frame}
    \section{Main features}
    \begin{frame}{General philosophy \& Code format}
        \alert{\textbf{Philosophy}}
        \begin{itemize}
            \item Write functions not scripts
            \item Julia has data types, but not objects
            \item write generic code “acting” on data
            \item no need to write “vectorized code”
            \item avoid global variables
        \end{itemize}
        \vspace{\baselineskip}
        \pause
        \alert{\textbf{Format}}
        \begin{itemize}
            \item blocks have an \mintinline{julia}|end|
            \item Indentation with 4 spaces is recommended but not necessary
            \item functions that modify their data should be named with an \mintinline{julia}|!|.
        \end{itemize}
    \end{frame}
    \begin{frame}[fragile]{Prequel: Packages \& Pluto Notebooks}
        A \alert{Package} is a \alert{module} (namespace) providing additional functionality.

        \begin{itemize}[<+->]
            \item To install one for our demos use the package mode
            \begin{minted}{julia-repl}
] add Pluto
            \end{minted}
            This has only to be done once.
            \item To load a package after starting Julia,
            use the \mintinline{julia}{using} keyword
            \begin{minted}{julia-repl}
using Pluto
            \end{minted}
            \item we can call a function from the package always by
            \begin{minted}{julia-repl}
Pluto.run()
            \end{minted}
            \item the last two can be done in one line, when using \mintinline{julia}|;| as a divider
            \begin{minted}{julia-repl}
using Pluto; Pluto.run()
            \end{minted}
        \end{itemize}
        \pause
        We will continue command demos in the \href{https://plutojl.org}{Pluto notebook}
        \\\hfill (similar to a Jupyter notebook, but with a persistent state)
    \end{frame}
    \begin{frame}[fragile]{Control flow I: for \& while}
        \begin{columns}[T]
            \begin{column}{.5\textwidth}
                \alert{Iterate} with for-loops
                \begin{minted}{julia}
for i=1:4
    print(i," ")
end # prints "1 2 3 4"
                \end{minted}
                \pause
                Combine several (and use $\in$)
                \begin{minted}[escapeinside=||,mathescape=true]{julia}
for i |$\in$| 1:3, j |$\in$| 1:2
    print(i,"×",j,", ")
end # prints 1$\times$1, 1$\times$2, ...
                \end{minted}
                \pause
                Or through several of same length
                \begin{minted}[escapeinside=||,mathescape=true]{julia}
for (i,j) |$\in$| zip(1:4, 5:8)
    print(i,"|",j," ")
end # prints 1|5 2|6 3|7 4|8
                \end{minted}
            \end{column}
            \pause
            \begin{column}{.5\textwidth}
                or as a \alert{comprehension} for vectors
                \begin{minted}[escapeinside=||,mathescape=true]{julia}
x = [3*s for s |$\in$| 1:3 ]
                \end{minted}
                creates \mintinline{julia}|[3, 6, 9]|
                \\[2.5\baselineskip]
                Loops with “unknown end”
                \begin{minted}{julia}
i = 1;
# do as long as i <= 4
while i <= 4
    print(i," ");
    i += 1
end # also prints "1 2 3 4"
                \end{minted}
            \end{column}
        \end{columns}
    \end{frame}
    \begin{frame}[fragile]{Control flow II: Conditionals}
        \alert{Conditionals} require an expression that evaluates to a \mintinline{julia}|Bool|. Then
        \begin{minted}{julia}
if (x > 3) || (z < 2) # brackets (x > 3) are optional
    print("x is at least 3")
else
    print("x is 3 or less")
end
        \end{minted}
        \pause
        There is \alert{lazy evaluation}: the second parts of
        \begin{minted}{julia}
 (x > 4) && print("x > 4")
(x <= 4) || print("x > 4")
        \end{minted}
        are only called/evaluated if $x > 4$.
        \\[\baselineskip]
        \pause
        Conditionals can be used inline with
        \begin{minted}{julia}
y = (x > 4) ? 1 : 3*x
        \end{minted}
    \end{frame}
    \begin{frame}[fragile]{Defining functions}
        \begin{minted}{julia}
"""
    phase(z)

Compute the phase of a complex number z
"""
function phase(z)
    return atan(imag(z), real(z))
end
        \end{minted}
        \begin{itemize}[<+->]
            \item \alert{naming convention} \mintinline{julia}|snake_case|
            \item (multiline) \mintinline{julia}|"String"| upfront: doc-string, may use  Markdown
            \item specify type with \mintinline{julia}|z::Number| (but \alert{avoid} overtyping like \mintinline{julia}{::Float64})
            \item (last) \mintinline{julia}|return| optional, but reommended\\
            \hfill(last evaluated expression returned)
        \end{itemize}
    \pause
    Shorter form
    \begin{minted}{julia}
magnitude(z) = sqrt(imag(z)^2+real(z)^2)
    \end{minted}
    \end{frame}
    \begin{frame}[fragile]{More on functions}
        \begin{itemize}[<+->]
            \item functions are first-class objects (like variables)
            \item anonymous function \mintinline{julia}|  (x,y) -> x^y  | e.g. to pass as parameter
            \item \alert{Broadcast}: apply \mintinline{julia}|phase(z)| to a whole vector
            \begin{minted}{julia}
Z = [1.0im, 2.0, 1.0 + 0.2im]
            \end{minted}
            by adding a \mintinline{julia}|.| after the function name: \mintinline{julia}|    phase.(Z)|
            \item broadcast with multiple vectors
            \begin{minted}{julia}
X = [0.1, 0.2, 0.3]; Y = [1.0, 2.0, 3.0]
X.^Y # same: [X[i]^Y[i] for i=1:3] or [0.1, 0.04, 0.027]
            \end{minted}
            \item functions can modify their input
            \begin{minted}{julia}
function add_scalar!(X, v)
    X .+= v  # X an array, v a scalar: add to every entry
    return X # the X we got passed is now changed
end
            \end{minted}
            \alert{Convention:} such a functions name ends in \mintinline{julia}|!|, it returns the modified
        \end{itemize}
    \end{frame}
    \begin{frame}[fragile]{Data structures}
%        Build a hierarchy with
%        \begin{minted}{julia}
%abstract type Landmarks end
%struct City <: Landmarks
%    name::String
%    position::Vector
%end
%        \end{minted}
%        \begin{itemize}[<+->]
%            \item the super type is implicit \mintinline{julia}|<:Any| if not specified
%            \item all fields (should) have a type
%            \item super types to build a hierarchy
%            \item \alert{naming convention:} \mintinline{juia}|CamelCase| for types
%        \end{itemize}
    \end{frame}
    \begin{frame}{Multiple Dispatch}
    \end{frame}
    \begin{frame}{Operators are Functions}
    \end{frame}
    \begin{frame}{Functors}
    \end{frame}
    \begin{frame}{Scripts}
    \end{frame}
    \begin{frame}[fragile]{TLDR: Main differences to Python}
        \begin{itemize}
            \item \mintinline{julia}|for, if, while| etc. blocks are terminated by  \mintinline{julia}|end|
            \item indentation is nice, but not mandatory
            \item \alert{Julia is 1-indexed}
            \item Strings have single \mintinline{julia}|"quotation marks"|, multiline strings three
            \pause
            \item loops amd vectors are fast (no need for vectorized code)
            \item abstract arrays allow arbitrary indexing $\Rightarrow$ \mintinline{python}|a[-1]| is in Julia \mintinline{julia}|a[end-1]|
            \item Julias range \mintinline{julia}|1:5| includes the end and has the general form \mintinline{julia}|start:step:stop| (instead of \mintinline{python}|start:(stop+1):step|)
            \item the imaginary unit is \mintinline{julia}|im| (not \mintinline{python}|j|)
            \pause
            \item Matrix multiplication is \mintinline{julia}|A * B|, element wise multiplication \mintinline{julia}|A .* B|
            \item Julia has no objects/classes
        \end{itemize}
    \end{frame}
    \begin{frame}[fragile]{TLDR: Main differences to R}
        \begin{itemize}
            \item \mintinline{julia}|‘single’| quotation marks are for characters
            \item vectors are constructed with square brackets \mintinline{julia}|v = [1,2,3]|
            \item operations on vectors of different length are not allowed
            \item \mintinline{julia}|<-|, \mintinline{julia}|<<-| and \mintinline{julia}|->| are not assignment operators
            \item \mintinline{julia}|->| creates an anonymous function
            \pause
            \item matrix multiplication is just \mintinline{julia}|A * B|
            \item function arguments are not copied when calling a function
            \item \mintinline{julia}|1:5| is an \mintinline{julia}|AbstractRange|, use \mintinline{julia}|collect(1:5)| to create the vector
            \pause
            \item you do not need vectorization for performance
            \item logical indexing: in R \mintinline{R}|x[x>3]| has two alternatives in Julia
            \begin{itemize}
                \item \mintinline{julia}|x[ x .> 3]| (uses a temporary vector memory)
                \item \mintinline{julia}|filter(z->z>3, x)| might be nicer to read
                \item \mintinline{julia}|filter!(z->z>3, x)| updates \mintinline{julia}!x! inplace (avoids the temporary memory)
            \end{itemize}
        \end{itemize}
    \end{frame}
    \begin{frame}[fragile]{TLDR: Main differences to Matlab}
        \begin{itemize}
            \item array indexing uses square brackets \mintinline{julia}|A[i,j]|
            \item Arrays are not copied by default \mintinline{julia}|A=B| references the same, do \mintinline{julia}|A=copy(B)| for an actual copy
            \item \emph{similarly} function arguments are references, \alert{input variables can be modified}
            \item 1-dimensional vectors exist and are not \mintinline{julia}|Nx1| matrices
            \item \mintinline{julia}|42| is an integer, not a float, use \mintinline{julia}|42.0| for the float.
            \item \mintinline{julia}|A == B| does not return a matrix of booleans but \mintinline{julia}|true| or \mintinline{julia}|false|\\
            use \mintinline{julia}|A .== B| to get such a matrix
            \item dimensions are not “constant-broadcasted”:\\
            \begin{itemize}
                \item \mintinline{matlab}|[1:10] + [1:10]'| creates a $10\times 10$ matrix in Matlab
                \item \mintinline{julia}|[1:10] + [1:10]'| is a dimension mismatch,\\
                 because a column vector can not be added to a row vector
            \end{itemize}
        \end{itemize}
    \end{frame}
    \section{Packages}
    \begin{frame}{Installing \& Using Pacakges}

    \end{frame}
    \begin{frame}{Package versions \& Updating}

    \end{frame}
    \begin{frame}{Package environments}

    \end{frame}
    \section{Pluto Notebooks}
    \begin{frame}{Pluto.jl – Motivation}
    \end{frame}
    \begin{frame}{Similarities \& differentes to Jupyter}
    \end{frame}
    \begin{frame}{Live Demo}
    \end{frame}
    \begin{frame}{Thanks for your attention!}
        Are there (further) questions?
    \end{frame}
    \section{Workshop: Let's get you started with Julia!}
\end{document}
