\documentclass[a5paper,12pt,DIV=14,footheight=-30pt, headheight=60pt, twoside=off]{scrartcl}
\usepackage[T1]{fontenc}
\usepackage[utf8]{inputenc}
\thispagestyle{empty}
\usepackage[inline]{enumitem}
\usepackage[english]{babel}
\usepackage{booktabs,listings,scrlayer-scrpage,xcolor}
\usepackage{mathsemantics, amsmath, amssymb, bm, hyperref}
\definecolor{juliared}{rgb}{0.796, 0.235, 0.2}
\definecolor{juliagreen}{rgb}{0.22, 0.596, 0.149}
\definecolor{juliablue}{rgb}{0.251, 0.388, 0.847}
\definecolor{juliapurple}{rgb}{0.584, 0.345, 0.698}
\setlength{\parskip}{.5\baselineskip}
\setlength{\parindent}{0pt}
\hypersetup{colorlinks=true,urlcolor=juliagreen, linkcolor=juliablue}
\setlist[enumerate,1]{itemsep=-.5ex}
\setlist[enumerate,2]{label=\color{black}\alph*)}
\newenvironment{sublist}[1][]{% begin code
  \color{juliapurple}%
  \quad\begin{enumerate*}[#1]%
}%
{\end{enumerate*}}
\newcommand{\pagenote}[1]{{\small({\color{juliablue}N\,#1})}}
\newcommand{\codenote}[1]{{\small({\color{juliablue}\texttt{#1}})}}
\newcommand{\timenote}[1]{\marginline{\tiny #1}}
\setkomafont{pageheadfoot}{\small}
%\setkomafont{pagehead}{\bfseries}
\lohead{Intro to Julia}
\rohead{}
%\cofoot{}
\begin{document}
\thispagestyle{empty}
\textbf{Intro to Julia\quad Vår 2025}\hfill{March 20, 2025, SB II, R5}\\
\begin{enumerate}
    \item Title:
    \textbf{Introduction to Julia}
    \item Overview
\end{enumerate}
\textbf{What is Julia}
\begin{enumerate}[resume]
    \item Goal: SciComp \& Prototyping
    \begin{sublist}
        \item high performance \item high-level \item Often C/C++
    \end{sublist}
    \item Combine both: Julia
    \begin{sublist}
        \item 3 main features \item history
    \end{sublist}
    \item Resources
\end{enumerate}
\textbf{Installation \& REPL}
\begin{enumerate}[resume]
    \item Installation
    \item REPL\\
    \emph{Have a first, empty terminal ready to start julia in}
    \\
    \begin{sublist}
        \item name \& examples (copy a few)
        \item shortcuts
    \end{sublist}
    \item Modes
    \begin{sublist}
        \item ? (illustrate with sqrt)\item ] \item ;
    \end{sublist}
\end{enumerate}
\textbf{Main features}
\begin{enumerate}[resume]
    \item General Philosophy \& format
    \begin{sublist}
        \item Philosophy \item Format
    \end{sublist}
    \item Prequel: Start a notebook\\
    \emph{Prepare that second terminal and the browser with Pluto open upfront}
    \\
    \begin{sublist}
        \item add \item using \item run \item or both together \item we continue
    \end{sublist}
    \item Control Flor I: for \& while
    \begin{sublist}
        \item for 1 \item for 2 \item for 3 \item compr \item while
    \end{sublist}
    \item Control Flow II: Conditionals
    \begin{sublist}
        \item if \item lazy \item inline
    \end{sublist}
    \item Defining functions
    \begin{sublist}
        \item naming \item doc \item type \item return \item shorter
    \end{sublist}
    \item More on functions I: args\&kwargs
    \begin{sublist}
        \item pos \item kwargs \item pass on
    \end{sublist}
    \item More on functions II: Broadcast \& Mutation
    \begin{sublist}
        \item first class \& anonymous \item broadcast \item multiple
        \item modify \item naming
    \end{sublist}
    \item Data structures
    \begin{sublist}
        \item abstract \item naming \item struct \item is immutable \item but fields
        \item efficient (2 artificial breaks!)
        \item mutable \item can change \item less
    \end{sublist}
    \item Parameyric types
    \begin{sublist}
        \item motivation \item II \item III \item def/clumsy \item nicer
        \item concrete vector \item own constructor
    \end{sublist}
    \item Multiple Dispatch
    \begin{sublist}
        \item best fitting \item Def \item we get \item most fitting
        \item multiple \item avoid \item resolve
    \end{sublist}
    \item Operators are functions
    \begin{sublist}
        \item Def \item then \item also parametric
    \end{sublist}
    \item Functors
    \begin{sublist}
        \item poly struct
        \item functor
        \item we get
    \end{sublist}
    \item Python
    \begin{sublist}
        \item end\item indent \item 1 \item string \item loops fast
        \item abstract arrays \item range \item imag \item matmul \item dispatch
    \end{sublist}
    \item R
    \begin{sublist}
        \item single quotes \item vectors \item ops on diff lengths \item assignment
        \item anon fct \item mat mul \item pass by ref \item abstract range
        \item no vectorize code
        \item indexing
    \end{sublist}
    \item Matlab
    \begin{sublist}
        \item arrays in [] \item copy arrays \item modify \item 1-dim
        \item int/float
        \item use broadcast
        \item no auto broadcast
    \end{sublist}
\end{enumerate}
\textbf{Packages}
\begin{enumerate}[resume]
    \item Namespaces \& Modules
    \begin{sublist}
        \item namespace
        \item use others
        \item export
        \item others with prefix
        \item ! name clash
        \item default packages
    \end{sublist}
    \item Installing \& using Packages
    \begin{sublist}
        \item Registy \& Pkg
        \item default
        \item package mode again
        \item add
        \item installs
        \item resolves versions
        \item status
        \item update
        \item using
    \end{sublist}
    \item Package environments
    \begin{sublist}
        \item env defined
        \item default
        \item activate
        \item current folder, easy!
        \item reproducible
        \item project
        \item manifest
        \item reproducible(!)
    \end{sublist}
\end{enumerate}
\textbf{Pluto Notebooks}
\begin{enumerate}[resume]
    \item Pluto.jl
    \begin{sublist}
        \item intro
        \item cells \& execute
        \item hide code
        \item live docs
        \item status
        \item start
    \end{sublist}
    \item Differences to Jupyter
    \begin{sublist}
        \item Script
        \item includable
        \item output
        \item git
        \item versions intern
        \item own env on start
        \item exact versions
        \item reproducible
        \item persistent state
        \item dependencies
        \item update all dependents
        \item global state
        \item no need to rememebr right order
    \end{sublist}
    \item Live Demo (create something with Makie and sliders?)
    \item Further topics
\end{enumerate}
\textbf{workshop!}
\end{document}
