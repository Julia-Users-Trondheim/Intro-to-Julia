%% main.tex
%% Copyright 2024 Níckolas de Aguiar Alves
%
% This work may be distributed and/or modified under the
% conditions of the LaTeX Project Public License, either version 1.3
% of this license or (at your option) any later version.
% The latest version of this license is in
%   https://www.latex-project.org/lppl.txt
% and version 1.3c or later is part of all distributions of LaTeX
% version 2008 or later.
%
% This work has the LPPL maintenance status `maintained'.
%
% The Current Maintainer of this work is Níckolas de Aguiar Alves.
%
% This work consists of the files main.tex, README.md
% and the derived file example.pdf.


% This a LaTeX template for a minimalist academic poster. It was inspired by the #betterposter designs by Mike Morrison (https://www.youtube.com/@MikeMorrisonPhD)
% This template was written by Níckolas de Aguiar Alves (alves-nickolas.github.io)

\documentclass[portrait]{a0poster} % a0poster class to get a portrait a0-sized poster
\usepackage{fontspec} %Please compile with LuaLaTeX to get advanced font functionalities. You can do this by going to the Menu (top-left of your screen, if you're on Overleaf) and selecting LuaLaTeX on Settings -> Compiler
\setmainfont{Merriweather Sans} % main font (I will use in the title)
\setsansfont{Carlito} % sans serif font (I will use elsewhere)
\setmonofont{Fira Mono} % monospaced font (just in case)
% the full set of fonts accepted by Overleaf is given in https://www.overleaf.com/learn/latex/Questions/Which_OTF_or_TTF_fonts_are_supported_via_fontspec%3F
\usepackage{microtype} % improves output
\usepackage{xcolor} % color capabilities
\usepackage{tikz} % for positioning elements
\usetikzlibrary{calc} % helps positioning elements
\usepackage{graphicx} % required for inserting images
\usepackage{qrcode} % automatically generated qr-codes from a link
\usepackage{hyperref} % hyperlinks in the digital version

\definecolor{background}{HTML}{FFFFFF} % color for the background
\definecolor{textcolor}{HTML}{000000} % color for the text
\definecolor{accent}{HTML}{f0be52} % highlight color
% define colors
\definecolor{NTNUBlue}{HTML}{00509e}
% support colors
\definecolor{NTNUgreen}{HTML}{bcd025}
\definecolor{NTNULightblue}{HTML}{6096d0}
\definecolor{NTNUOrange}{HTML}{ef8114}
\definecolor{NTNUPink}{HTML}{b01b81}
\definecolor{NTNUYellow}{HTML}{f7d019}
\definecolor{NTNUViolet}{HTML}{482776}
\definecolor{NTNUCyan}{HTML}{3cbfbe}
\definecolor{NTNUOcher}{HTML}{cfb887}
\definecolor{LightGrey}{HTML}{bebebe}
\definecolor{juliared}{rgb}{0.796, 0.235, 0.2}
\definecolor{juliagreen}{rgb}{0.22, 0.596, 0.149}
\definecolor{juliablue}{rgb}{0.251, 0.388, 0.847}
\definecolor{juliapurple}{rgb}{0.584, 0.345, 0.698}
\hypersetup{colorlinks=true, urlcolor=juliapurple}
\newcommand{\hl}[1]{\textcolor{juliablue}{#1}} % shortcut for highlighting

\begin{document}
\begin{tikzpicture}[remember picture, overlay, shift={(current page.center)}] % starts a tikz picture environment centered at the center of the page
    \fill[background] (current page.south east) rectangle (current page.north west); % fills a rectangle covering the whole page with the background color

    \node[textcolor,font={\VeryHuge\bfseries},scale=2.25,text width=0.41\pdfpagewidth,align=left] (title) at (0.2,5cm) {\hl{Introduction to} \raisebox{-.18\height}{\includegraphics[height=1.1\baselineskip]{../img/julia-logo-color.png}}\\\rightline{}}; % comically large font for the title (\VeryHuge + scale=2). The text width is chosen so that it occupies 82% (remember the scale=2) of the total page width (I chose this number because I liked it). Left aligned because I thought it looks better, but I throw in a \rightline on the bottom line to get the final output. The title is centered at (0,5cm), so centered horizontally, but a bit closer to the top of the page than to the bottom. The \hl command is used to hightlight some keywords

    \node[anchor=north east,textcolor,font={\huge},align=right,scale=2] at ($(title.south east)+(0,5cm)$) {Presentation \& Workshop}; % subtitle. Scale=2 is necessary to ensure alignment with the title. The subtitle is written so that its right side is aligned to the title's right side. Line breaks add some charm and I think they look better if they form a decreasing sequence from the title downward (I modified lorem ipsum to force this effect, and you also may need to rephrase some things to get this to work)

    \node[textcolor,font={\huge\bfseries},scale=2,text width=0.41\pdfpagewidth,align=left] (place) at (0.2,-15cm) {March 20, 2025, 16.00.\\\rightline{%Seminar room
    S5, SB\,II, NTNU.}}; % comically large font for the title (\VeryHuge + scale=2). The text width is chosen so that it occupies 82% (remember the scale=2) of the total page width (I chose this number because I liked it). Left aligned because I thought it looks better, but I throw in a \rightline on the bottom line to get the final output. The title is centered at (0,5cm), so centered horizontally, but a bit closer to the top of the page than to the bottom. The \hl command is used to hightlight some keywords

    \coordinate (bottomline) at ($(current page.south west)+(0,0.09\pdfpagewidth)$); % coordinate to mark the bottom of the ``usable page''. I won't add any elements below this point

    \coordinate (topline) at ($(current page.north east)+(0,-0.11\pdfpagewidth)$); % similar to bottomline

    \node[background,anchor=south east] (qrcodebase) at (title.east |- bottomline) {\qrcode[nolink,height=0.225\pdfpagewidth]{https://julia-users-trondheim.github.io}}; % auxiliary qr code to establish the dimensions of the actual qr code. This is used to draw a white background precisely where the qr code is located, so that it is actually visible

    \fill[textcolor] (qrcodebase.north west) rectangle (qrcodebase.south east); % white background for qr code

    \node[background,anchor=south east] at (qrcodebase.south east) {\qrcode[nolink,height=0.225\pdfpagewidth]{https://julia-users-trondheim.github.io}}; % automatically generated qr code, correctly aligned with the title. Replace www.google.com with the link you want to use: the longer the link, the more complex (and uglier) the qr code. The qr code is a square with size equal to 22.5% of the page width (this worked for me). Since its size is used to establish the alignment of the text next to it, you may want to play around a bit to get it to work.

    \draw (title.west |- qrcodebase.north) node[anchor=north west,text width=0.6\pdfpagewidth,textcolor,align=justify,font={\huge\sffamily}] (parone) {%
    \hl{Julia} is an open source programming language with scientific computing as one of its main targets. It allows for \hl{fast prototyping} while also directly producing \hl{fast}, \hl{high performance} as well as \hl{reproducible code}.
    \\[\baselineskip]
In this workshop we give an introduction to the language, how to install \hl{Julia}, get started and work with \hl{Pluto} notebooks.
A main focus is based on how to start programming in \hl{Julia} if you come from Python, Matlab or R.
Afterwards we will help you get started, work on \hl{concrete ideas} and \hl{first small projects}.
};
    \draw (title.east |- bottomline) node[anchor=north east,text width=0.45\pdfpagewidth,textcolor,align=justify,font={\LARGE\sffamily}] (paragraph) {\rightline{\href{https://julia-users-trondheim.github.io}{julia-users-trondheim.github.io}.}}; % reference for your work

    \draw (title.east |- topline) node[anchor=east] (logos) {\includegraphics[height=0.25\pdfpagewidth]{../img/logo.png}}; % university logos, funding agencies, etc
\end{tikzpicture}
\end{document}